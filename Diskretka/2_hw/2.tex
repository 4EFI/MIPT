\documentclass[a4paper, 12pt]{article}
\usepackage{geometry}
\geometry{a4paper,
	total={170mm,257mm},left=2cm,right=2cm,
	top=1cm,bottom=2cm}

\usepackage{mathtext}
\usepackage{amsfonts} 
\usepackage{amsmath}
\usepackage[T2A]{fontenc}
\usepackage[utf8]{inputenc}
\usepackage[english,russian]{babel}
\usepackage{graphicx, float}
\usepackage{tabularx, colortbl}
\usepackage{caption}
\captionsetup{labelsep=period}

\newcommand{\parag}[1]{\paragraph*{#1:}}
\DeclareSymbolFont{T2Aletters}{T2A}{cmr}{m}{it}
\newcounter{Points}
\setcounter{Points}{1}
\newcommand{\point}{\arabic{Points}. \addtocounter{Points}{1}}
\newcolumntype{C}{>{\centering\arraybackslash}X}

\author{Костылев Влад, Б01-208}
\date{\today}
\title{\textbf{Алгебра логики: введение} \\ 
	домашнее задание}

\begin{document}
	\maketitle
	
	%1
	\textbf{№1}\
	\
	\[
		\neg(x = y) \wedge ((y < x) \rightarrow (2z > x)) \wedge ((x < y) \rightarrow (x > 2z)) 
	\]
	\[
		\Rightarrow \
		(x \not= y) \wedge ((y \geq x) \vee (2z > x)) \wedge ((x \geq y) \vee (x > 2z))
	\]
	\[
		\Leftrightarrow \
		\left\{ 
		\begin{gathered} 
			x \not= 16; \\
			\left[
			\begin{gathered}
				x \leq 16; \\ 
				x < 14     \\
			\end{gathered} 
			\right.
			\\
			\left[
			\begin{gathered}
				x \geq 16; \\ 
				x > 14.    \\
			\end{gathered} 
			\right.
		\end{gathered} 
		\right. \
		\Longrightarrow \
		x = 15
	\]
	
	%2
	\textbf{№2}\
	\
	\begin{table}[H]
	\centering	
	\begin{tabular}{|c|c|c|c|c|c|}
		\hline
		x & y & z & $\neg y$ & $x \wedge \neg y$ & $\neg((x \wedge \neg y) \wedge z)$\\ \hline
		0 & 0 & 0 & 1 & 0 & 1 \\ \hline
		0 & 0 & 1 & 1 & 0 & 1 \\ \hline
		0 & 1 & 0 & 0 & 0 & 1 \\ \hline
		0 & 1 & 1 & 0 & 0 & 1 \\ \hline	
		1 & 0 & 0 & 1 & 1 & 1 \\ \hline
		1 & 0 & 1 & 1 & 1 & 0 \\ \hline
		1 & 1 & 0 & 0 & 0 & 1 \\ \hline
	    1 & 1 & 1 & 0 & 0 & 1 \\ \hline
	\end{tabular}
	\end{table}

	%3
	\textbf{№3}\
	\
	\[
		1 \oplus x_1 \oplus x_2 = \neg x_1 \oplus x_2 = x_1 \oplus \neg x_2 = (x_1 \leftrightarrow x_2)
	\]
	\[
		(x_1 \rightarrow x_2) \wedge (x_2 \rightarrow x_1) = (\overline{x_1} \vee x_2) \wedge (\overline{x_2} \vee x_1) = (x_1 \leftrightarrow x_2) \Rightarrow ч.т.д.
	\]
	
	%4
	\textbf{№4}\
	\
	\textbf{a)} $x \wedge (y \rightarrow z) \stackrel{?}{=} (x \wedge y) \rightarrow (x \wedge z)$
	\[
		x \wedge (y \rightarrow z) = x \wedge (\overline{y} \vee z) = (x \wedge \overline{y}) \vee (x \wedge z) 
	\]
	\[
		(x \wedge y) \rightarrow (x \wedge z) = (\overline{x \wedge y}) \vee (x \wedge z) \Rightarrow неверно
	\]
	\
	\textbf{b)} $x \oplus (y \leftrightarrow z) \stackrel{?}{=} (x \oplus y) \leftrightarrow (x \oplus z)$
	\[
		x \oplus (y \leftrightarrow z) = x \oplus \overline{y} \oplus z
	\]
	\[
		(x \oplus y) \leftrightarrow (x \oplus z) = (\overline{x \oplus y}) \oplus (x \oplus z) = (\overline{y} \oplus x) \oplus (x \oplus z)	= \overline{y} \oplus z \Rightarrow неверно 
	\]
			
	%5
	\textbf{№5}\
	\
	\textbf{a)} $x \rightarrow y \stackrel{?}{=} y \rightarrow x$
	\begin{table}[H]
		\centering	
		\begin{tabular}{|c|c|c|c|c|}
			\hline
			x & y & $x \rightarrow y$ & $y \rightarrow x$ \\ \hline
 			0 & 0 & 1 & 1 \\ \hline
			0 & 1 & 1 & 0 \\ \hline
			1 & 0 & 0 & 1 \\ \hline
			1 & 1 & 1 & 1 \\ \hline	
		\end{tabular}
	\end{table}
	\[
		\Rightarrow неверно
	\]
	\
	\textbf{b)} $(x \rightarrow y) \rightarrow z \stackrel{?}{=} x \rightarrow (y \rightarrow z)$
	
		\begin{table}[H]
		\centering	
		\begin{tabular}{|c|c|c|c|c|c|}
			\hline
			x & y & z & $x \rightarrow y$ & $(x \rightarrow y) \rightarrow z$ \\ \hline
			0 & 0 & 0 & 1 & 0 \\ \hline
			0 & 0 & 1 & 1 & 1 \\ \hline
			0 & 1 & 0 & 1 & 0 \\ \hline
			0 & 1 & 1 & 1 & 1 \\ \hline	
			1 & 0 & 0 & 0 & 1 \\ \hline
			1 & 0 & 1 & 0 & 1 \\ \hline
			1 & 1 & 0 & 1 & 0 \\ \hline
			1 & 1 & 1 & 1 & 1 \\ \hline
		\end{tabular}
		\
		\begin{tabular}{|c|c|}
			\hline
			$y \rightarrow z$ & $x \rightarrow (y \rightarrow z)$ \\ \hline
			1 & 1 \\ \hline
			1 & 1 \\ \hline
			0 & 1 \\ \hline
			1 & 1 \\ \hline	
			1 & 1 \\ \hline
			1 & 1 \\ \hline
			0 & 0 \\ \hline
			1 & 1 \\ \hline
		\end{tabular}
	\end{table}
	\[
	\Rightarrow неверно
	\]
	
	%6
	\textbf{№6}\
	\
	\textbf{a)} $f(x_1,x_2, x_3) = 00111100$
	\begin{table}[H]
	\centering	
	\begin{tabular}{|c|c|c|c|c|c|}
		\hline
		$x_1$ & $x_2$ & $x_3$ & $f(x_1,x_2, x_3)$ \\ \hline
		0 & 0 & 0 & 0 \\ \hline
		0 & 0 & 1 & 0 \\ \hline
		0 & 1 & 0 & 1 \\ \hline
		0 & 1 & 1 & 1 \\ \hline	
		1 & 0 & 0 & 1 \\ \hline
		1 & 0 & 1 & 1 \\ \hline
		1 & 1 & 0 & 0 \\ \hline
		1 & 1 & 1 & 0 \\ \hline
	\end{tabular}
	\
	$\Rightarrow x_1, x_2 - \text{существенные}, x_3 - \text{фиктивная,}$
	\\
	т.к. $f(0, x_2, x_3) \not= f(1, x_2, x_3)$ и $f(x_1, 0, x_3) \not= f(x_1, 1, x_3)$
	\end{table}
	\
	\textbf{b)} $g(x_1, x_2, x_3) = (x_1 \rightarrow (x_1 \vee x_2)) \rightarrow x_3$
	\[
		(x_1 \rightarrow (x_1 \vee x_2)) \rightarrow x_3 = (\overline{x_1} \vee x_1 \vee x_2) \rightarrow x_3 = 1 \rightarrow x_3 = 0 \vee x_3 = x_3
	\]
	$\Rightarrow$ $x_1, x_2 - \text{фиктивные}$, $x_3 - \text{существенная}$ 
	
	%7
	\textbf{№7}\
	\
	Доказать: $f(x_1, \dots, x_n) = (x_1 \vee f(0, x_2, \dots, x_n)) \wedge (\overline{x_1} \vee f(1, x_2, \dots, x_n))$
	\\
	$x_1 = 0 \Rightarrow$
	\[
		f(0, x_2, \dots, x_n) = (0 \vee f(0, x_2, \dots, x_n)) \wedge \overbrace{(1 \vee f(1, x_2, \dots, x_n))}^{1} = (0 \vee f(0, x_2, \dots, x_n)) = f(0, x_2, \dots, x_n)
	\]
	$X_1 = 1 \Rightarrow$
	\[
		f(1, x_2, \dots, x_n) = \overbrace{(1 \vee f(0, x_2, \dots, x_n))}^{1} \wedge (0 \vee f(1, x_2, \dots, x_n)) = (0 \vee f(1, x_2, \dots, x_n)) = f(1, x_2, \dots, x_n)	
	\]
	$\Rightarrow$ ч.т.д.
	
	%8
	\textbf{№8}\
	\
	Функция истинна $\forall i \in \mathbb{N}: x_{i}^{\alpha_i} = 1$. Если будет хотя бы один 0, то функция будет ложна (набор одних конъюнкций) $\Rightarrow$ т.к. $x_1, x_2, \dots, x_n$ - это некий фиксированный набор, то:
	\[
		\forall i \in \mathbb{N}, i \leq n: \alpha_i =
		\left\{
			\begin{gathered}
			1, x_i = 1; \\
			0, x_i = 0; \\
			\end{gathered}
		\right. 
	\]
	$\Rightarrow$ набор $\alpha_1, \alpha_2, \dots, \alpha_n$ фиксирован, ч.т.д. 
	
	
			
\end{document}