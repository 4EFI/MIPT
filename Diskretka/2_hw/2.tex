\documentclass[a4paper, 12pt]{article}
\usepackage{geometry}
\geometry{a4paper,
	total={170mm,257mm},left=2cm,right=2cm,
	top=1cm,bottom=2cm}

\usepackage{mathtext}
\usepackage{amsmath}
\usepackage[T2A]{fontenc}
\usepackage[utf8]{inputenc}
\usepackage[english,russian]{babel}
\usepackage{graphicx, float}
\usepackage{tabularx, colortbl}
\usepackage{caption}
\captionsetup{labelsep=period}

\newcommand{\parag}[1]{\paragraph*{#1:}}
\DeclareSymbolFont{T2Aletters}{T2A}{cmr}{m}{it}
\newcounter{Points}
\setcounter{Points}{1}
\newcommand{\point}{\arabic{Points}. \addtocounter{Points}{1}}
\newcolumntype{C}{>{\centering\arraybackslash}X}

\author{Костылев Влад, Б01-208}
\date{\today}
\title{\textbf{Алгебра логики: введение} \\ 
	домашнее задание}

\begin{document}
	\maketitle
	
	%1
	\textbf{№1}\
	\
	\[
		\neg(x = y) \wedge ((y < x) \rightarrow (2z > x)) \wedge ((x < y) \rightarrow (x > 2z)) 
	\]
	\[
		\Rightarrow \
		(x \not= y) \wedge ((y \geq x) \vee (2z > x)) \wedge ((x \geq y) \vee (x > 2z))
	\]
	\[
		\Leftrightarrow \
		\left\{ 
		\begin{gathered} 
			x \not= 16; \\
			\left[
			\begin{gathered}
				y \leq 16; \\ 
				x < 14     \\
			\end{gathered} 
			\right.
			\\
			\left[
			\begin{gathered}
				y \leq 16; \\ 
				x < 14.    \\
			\end{gathered} 
			\right.
		\end{gathered} 
		\right. \
		\Longrightarrow \
		x = 15
	\]
	
	%2
	\textbf{№2}\
	\
	\[
		\neg((x \wedge \neg y) \wedge z)
	\]
	
	\begin{table}[H]
	\centering	
	\begin{tabular}{|c|c|c|}
		\hline
		x & y & z \\ \hline
		0 & 0 & 0 \\ \hline
		0 & 0 & 1 \\ \hline
		0 & 1 & 0 \\ \hline
		0 & 1 & 1 \\ \hline	
		1 & 0 & 0 \\ \hline
		1 & 0 & 1 \\ \hline
		1 & 1 & 0 \\ \hline
	    1 & 1 & 1 \\ \hline
	\end{tabular}
	\end{table}
			
\end{document}